\documentclass{docist}

\usepackage[latin1]{inputenc}
\usepackage[noanswers]{exameIST}

\input{macros}

\cabecalho{{\LARGE Software Architecture}\\
  {\large Mestrado em Engenharia Inform�tica e de Computadores}}


\lhead{}
 

\begin{document}

\thispagestyle{empty}

\begin{center}
  {\Large \textsc{Companion to the second exam on January 31st, 2017\\[2ex]
  \textbf{Version: A}\\[2ex]}}
  {\normalsize  \textbf{(You do not need to turn in this set of pages with your exam)}}
\end{center}


%1
\newcommand{\qFeaturitis}{
\begin{ClosedQuestion}
	Frank Buschmann states that:
		
	\begin{quote}
		Featuritis is the tendency to trade functional coverage for quality - the more functions the earlier they're delivered, the better.
	\end{quote}

    \optionA{Featuritis may result from a requirement of the technical context.}
    \optionB{Featuritis requires the performance quality because the end user needs to execute the features.}
    \optionC{Featuritis may be a result of a requirement of the business context.}
    \optionD{Featuritis requires the modifiability quality to allow a the system to be easily modified to support new features.}
 \putOptions 
\end{ClosedQuestion}
}

%2
\newcommand{\qExplicit}{
\begin{ClosedQuestion}
	Frank Buschmann states that:
	
	\begin{quote}
		There's only one escape from such situations: architects must actively break the cycle of mutual misunderstanding and mistrust!
	\end{quote}

    \optionA{Such misunderstanding and mistrust occurs because the stakeholders have their own agendas}
    \optionB{The cycle Frank Buschmann refers to is the Architectural Influence Cycle.}
    \optionC{The cycle Frank Buschmann refers to allows the clarification of requirements.}
    \optionD{To break such misunderstanding and mistrust the architecture has to make explicit the stakeholders needs.}
 \putOptions
\end{ClosedQuestion}
}

%3
\newcommand{\qFlexibilitis}{
\begin{ClosedQuestion}
	Frank Buschmann states that:
	
	\begin{quote}
		Overly flexible systems are hard to configure, and when they're finally configured, they lack qualities like performance or security.
	\end{quote}

    \optionA{Frank Buschmann is referring to some possible consequences of the modifiability quality.}
    \optionB{Frank Buschmann are considering performance and security as the most important qualities.}
    \optionC{Frank Buschmann is referring that the consequences of a flexible system is poor performance and bad security.}
    \optionD{Frank Buschmann is not considering modifiability as an important quality}
 \putOptions
\end{ClosedQuestion}
}

%4
\newcommand{\qPrioritize}{
\begin{ClosedQuestion}
	Frank Buschmann states that:
	
	\begin{quote}
		Architects use flexibility as a cover for uncertainty.
	\end{quote}

    \optionA{A flexible architecture occurs when it is not possible to identify all the requirements.}
    \optionB{A solution to this problem is to prioritize the system qualities.}
    \optionC{Performance uncertainty about the system should be dealt with more flexibility.}
    \optionD{A solution to this problem is to reduce the level of flexibility of a system.}
 \putOptions
\end{ClosedQuestion}
}

%5
\newcommand{\qPerformitis}{
\begin{ClosedQuestion}
	Frank Buschmann cites the characterization Marquardt does of Performitis:
		
	\begin{quote}
		Each part of the system is directly influenced by local performance tuning measures. There is no global performance strategy, or it ignores other qualities of the system as testability and maintainability.
	\end{quote}
	
	From this problem you can conclude that:

    \optionA{Performance is a quality that you have to address at the end of the development process.}
    \optionB{There is no system which can have good performance and be easily maintainable.}
    \optionC{We have to distinguish architectural performance from opportunistic performance.}
    \optionD{The system performance quality has impact on the performance of the execution of tests.}
 \putOptions
\end{ClosedQuestion}
}

%6
\newcommand{\qFeaturitisPerformitisFlexibilities}{
\begin{ClosedQuestion}
	In his article, \emph{Featuritis, Performitis, and Other Deseases}, Frank Buschmann claims that:

    \optionA{Performance should be the last quality to be addressed because it is a local property of an architecture.}
    \optionB{Modifiability, flexibility, should be the first quality to be addressed because it allows the delay of architectural decisions.}
    \optionC{The lack of functionality results in a system without business value, therefore a rich set of features should be implemented first.}
    \optionD{A solution for any quality in isolation may lead to a biased architecture.}
 \putOptions
\end{ClosedQuestion}
}

%7
\newcommand{\qWalkingSkeleton}{
\begin{ClosedQuestion}
	The \emph{Walking Skeleton} referred in Frank Buschmann's article, \emph{Featuritis, Performitis, and Other Deseases}:

    \optionA{Is a functional prototype, which tests the functionalities required by the business stakeholders.}
    \optionB{Is an architecture that demonstrates that the system will support the qualities raised by the stakeholders.}
    \optionC{Is a baseline architecture that allows to experiment with the most significant architectural requirements.}
    \optionD{Is an object-oriented framework, which integrates functional and non-functional requirements of the system.}
 \putOptions
\end{ClosedQuestion}
}

%8
\newcommand{\qHammersNails}{
\begin{ClosedQuestion}
	In his article \emph{On Hammers and Nails, and Falling in Love with Technology and Design} what is the main type of influence on the architecture?

    \optionA{Project and Technical Contexts.}
    \optionB{Project and Professional Contexts.}
    \optionC{Business and Project Contexts.}
    \optionD{Professional and Technical Contexts.}
 \putOptions
\end{ClosedQuestion}
}

%9
\newcommand{\qArchitectureDefinition}{
\begin{ClosedQuestion}
	On the course slides you can find the following definition of architecture:
	
	\begin{quote}
		The software architecture of a program or computing system is the structure or structures of the system, which comprise software elements, the externally visible properties of those elements, and the relationships among them.
	\end{quote}
	
	However, in the book you can find another definition:
	
	\begin{quote}
		The software architecture of a system is the set of structures needed to reason about the system, which comprise the software elements, relations among them, and the properties of both.
	\end{quote}

    \optionA{The book definition does not consider relevant the externally visible properties.}
    \optionB{The book definition also considers that the properties are externally visible because they are used for reasoning by the stakeholders.}
    \optionC{The book definition also considers that the properties are externally visible because by definition an architectural property is externally visible.}
    \optionD{The book definition is not correct, as pointed out in the errata.}
 \putOptions
\end{ClosedQuestion}
}

%10
\newcommand{\qComponentvsModule}{
\begin{ClosedQuestion}
	In wikipedia you can find the following fragment of a definition:
	
	\begin{quote}
		An individual software component is a software package, or a module that encapsulates a set of related functions.
	\end{quote}
	
	According to the definitions taught in the course the above \emph{individual software component} corresponds to:

    \optionA{A component.}
    \optionB{A module.}
    \optionC{Both, a component and a module, depending on the perspective.}
    \optionD{An external element.}
 \putOptions
\end{ClosedQuestion}
}

%11
\newcommand{\qComponentvsModuleTwo}{
\begin{ClosedQuestion}
	In the Java documentation you can find:
	
\begin{quote}
\texttt{public abstract class Component} \\*
\texttt{extends Object} \\*
\texttt{implements ImageObserver, MenuContainer, Serializable}
\end{quote}

	Class \texttt{Component} is:

    \optionA{A component.}
    \optionB{A module.}
    \optionC{Both, a component and a module, depending on the perspective.}
    \optionD{An external element.}
 \putOptions
\end{ClosedQuestion}
}

%12
\newcommand{\qFunctionalModule}{
\begin{ClosedQuestion}
	When designing an architecture requirements can be split into functional, quality attributes, and constraints. Functional requirements have impact on:
		
    \optionA{A module view.}
    \optionB{A component-and-connector view.}
    \optionC{An allocation view.}
    \optionD{They are not represented by a view.}
 \putOptions
\end{ClosedQuestion}
}

%13
\newcommand{\qModuleViewType}{
\begin{ClosedQuestion}
	The quality that is more relevant to views of the module viewtype is:
		
    \optionA{Modifiability.}
    \optionB{Usability.}
    \optionC{Security.}
    \optionD{Availability.}
 \putOptions
\end{ClosedQuestion}
}

%14
\newcommand{\qComponentViewType}{
\begin{ClosedQuestion}
	The quality(ies) that is(are) more relevant to views of the component-and-connector viewtype is(are):
		
    \optionA{Modifiability.}
    \optionB{Availability and Performance.}
    \optionC{Testability.}
    \optionD{Availability.}
 \putOptions
\end{ClosedQuestion}
}

%15
\newcommand{\qEarlyDecisions}{
\begin{ClosedQuestion}
	In his article, \emph{Who Needs and Architect?}, Martin Fowler cites Ralph Johnson definition:
	
	\begin{quote}
		Architecture is the set of decisions that must be made early in a project.
	\end{quote}
	
	In his opinion:
		
    \optionA{This is right because if you don't the project fails.}
    \optionB{This is wrong because you can easily change these decisions during the project lifetime.}
    \optionC{This is right but you cannot be completely sure whether the decisions are the right ones.}
    \optionD{This is wrong because it is against agile way of thinking the software development process.}
 \putOptions
\end{ClosedQuestion}
}

%16
\newcommand{\qSharedUnderstanding}{
\begin{ClosedQuestion}
	Martin Fowler, \emph{Who Needs and Architect?}, cites Ralph Johnson sentence:
	
	\begin{quote}
		In most successful software projects, the expert developers working on that project have a shared understanding of the system design. This shared understanding is called architecture.
	\end{quote}
			
    \optionA{This shared understanding is what distinguishes architecture from design.}
    \optionB{This shared understanding is necessary to define precise requirements.}
    \optionC{This shared understanding does not allow to define the architecture trade-offs because some of the stakeholders have their own goals.}
    \optionD{This shared understanding does not allow to have a global perspective of the system, because stakeholders have different interests.}
 \putOptions
\end{ClosedQuestion}
}

%17
\newcommand{\qArchitectDwarves}{
\begin{ClosedQuestion}
	Frank Buschmann, \emph{Introducing the Pragmatic Architect}, defines the \emph{architecture dwarves}. These kind of architects
	
    \optionA{Are unable to understand the technology capabilities.}
    \optionB{Are focused on the project context of the architecture.}
    \optionC{Are unable to distinguish architecture from design.}
    \optionD{Are focused on the business context of the architecture.}
 \putOptions
\end{ClosedQuestion}
}

%18
\newcommand{\qArchitectAstronauts}{
\begin{ClosedQuestion}
	Frank Buschmann, \emph{Introducing the Pragmatic Architect}, defines the \emph{architecture astronauts}. This kind of architect
	
    \optionA{Is unable to define a domain model of the system.}
    \optionB{Is focused on the technology context of the architecture.}
    \optionC{Is focused on creating common generalizations of several systems.}
    \optionD{Is focused on the details of the architecture.}
 \putOptions
\end{ClosedQuestion}
}

%19
\newcommand{\qCreateArchitectureOne}{
\begin{ClosedQuestion}
	During the different steps on how to create an architecture, the precise specification of architecture quality attributes is initially relevant to
	
    \optionA{Make a business case for the system.}
    \optionB{Understand the architecturally significant requirements.}
    \optionC{The system design.}
    \optionD{Documenting and communicating the architecture.}
 \putOptions
\end{ClosedQuestion}
}

%20
\newcommand{\qCreateArchitectureTwo}{
\begin{ClosedQuestion}
	The \emph{Ensuring that the implementation conforms to the architecture} step of how to create an architecture
	
    \optionA{Tries to guarantee that the final system will have the qualities required by stakeholders.}
    \optionB{Tries to guarantee that the final system will have the qualities aimed by the architecture.}
    \optionC{Does not allow developers to define some of the design of the system}
    \optionD{It requires automatic generation of code from the architecture.}
 \putOptions
\end{ClosedQuestion}
}







% Software Architecture


\qSoftwareArchitectureTwo

% General scenarios 


\qConcreteScenarios

% Availability scenario


\qAvailabilityScenarioTwo

% Availability tactic


\qAvailabilityINGLES

% Performance scenario


\qInfinispanThree

% Performance tactic


\qPerfomanceTacticTwo

% Modifiablity scenario


\qModifiabilityScenario

% Modifiablity tactic


\qAspectsTactics

% Hadhoop - Scenario


\qHadoopStakeholdersEurosINGLES

% Hadhoop - Tactic


\qHadoopHeartbeatINGLES

% Architectural Views


\qArchitecturalViews

% Module viewtype


\qFunctionalModule

% Module architectural style one


\qSubcontractorsINGLES

% Module architectural style two


\qUsesStyle

% Module architectural style three


\qnginxModuleTypesINGLES

% Adventure Builder One


\qAdventureBuilderFour

% Adventure Builder Two


\qAdventureBuilderSix

% Catalog of DVDs


\qDVDCatalogMobile

% Component-and-connector viewtypes


\qTiposVistaDesempenhoINGLES

% Component-and-connector style one


\qPeerToPeerSpace

% Component-and-connector style two


\qTresTiersINGLES

% Component-and-connector style three


\qChromeMultiPlatform

% Twitter - Timelines of Scale - Scenario


\qSOAInteroperability

% Component-and-connector style four


\qInstallView

% Component-and-connector style five


\qTwitterTwo

% Twitter - Timelines of Scale - Tactic


\qTwitterFour

% Microservices


\qMicroservicesTwo

% Amazon architecture


\qWorldWideEN

% Boundaded Contexts and transactional architecture


\qBoundedContextTwo

% Effective aggregate design


\qDomainDesignTwo

\end{document}


